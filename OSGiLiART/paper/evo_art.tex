% !TEX encoding = UTF-8 Unicode
\documentclass{llncs}
\usepackage{llncsdoc}
\usepackage[T1]{fontenc}
\usepackage[utf8]{inputenc}
\usepackage{geometry}                % See geometry.pdf to learn the layout options. There are lots.
\geometry{letterpaper}                   % ... or a4paper or a5paper or ... 
%\geometry{landscape}                % Activate for for rotated page geometry
%\usepackage[parfill]{parskip}    % Activate to begin paragraphs with an empty line rather than an indent
\usepackage{graphicx}
\usepackage{amssymb}
\usepackage{epstopdf}
\DeclareGraphicsRule{.tif}{png}{.png}{`convert #1 `dirname #1`/`basename #1 .tif`.png}

\title{Evolutionary Art}
\author{}
%\date{}                                           % Activate to display a given date or no date

\begin{document}
\maketitle

\begin{abstract}
<Text of the summary of your article>
\end{abstract}

\section{Introduction}\label{intro}

This paper is organized as follows: In the next Section, a brief review on Evolutionary Art is presented. 

\section{Evolutionary Art}\label{evo_art}

Creative evolutionary systems are used to evolve aesthetically pleasing or innovative structures \cite{dipaola2009incorporating}.

\subsection{Art Representation for Evolutive Art} \label{evo_art:repr}
\begin{itemize}
	\item alalala
\end{itemize}
%\subsection{}

\subsection{Aesthetic measures for evolutive art}\label{evo_art:aesth}
MAIN CHALLENGE -> HOW TO MEASURE AESTHETICS.

{\bf Definition} Two modes of aesthetics measures can be defined \cite{galanter2012computational}: 

\begin{enumerate}
\item {\em Aesthetics evaluations are expected to simulate, predict or cater to humans notions of beauty and taste.} 
\item {\em Is an aspect of meta-aesthetic exploration and usually involves aesthetic standards created by software agents in artificial worlds.}
\end{enumerate}

According to Galanter \cite{galanter2012computational}, computational aesthetics measures can be classified in the following categories:
\begin{itemize}
	%\item Based on Formulaic and Geometric Theories. The aesthetics of a piece of art are evaluated using a formula o principle (e.g., pythagorean proportions).
	\item Based in Design Principles. Like the rule of thirds or theory of color (e.g., opposite colors) \cite{den2012evolving}.
	\item Based in Neural Networks and Connective Models. 
	\item Based in Evolutionary Systems:
		\begin{itemize}
			\item Interactive Evolutionary Computation. The fitness of the individuals is determined by human agents.
			\item Performance based goals. Certain properties of the art piece are evaluated and optimized based in performance measures (e.g., usable surface in furniture design generator). %An example of this approach is presented in  
			\item Error relative to Exemplars. The individual fitness is measured using a real-world example (e.g., a photography or painting) \cite{dipaola2009incorporating}.
			\item Complexity measures. This type of measures is based in the idea the complexity is directly related to aesthetics, following the path firstly stablished by Birkhoff \cite{Birkhoff:1933fk}.
			\item Multi-objective. Given the multidimensional nature of aesthetics judgement, multi-objective EAs are a clear option in order to deal with this multidimensionality.
			\item Extensions to EA (such as, coevolution, agent swarm behavior, etc.).
		\end{itemize}
	\item Complexity Based Models
\end{itemize}

%PAPERS LEÍDOS
En \cite{li2012investigating}, Li et al. proponen las siguiente métricas para el aprendizaje estético:
\begin{itemize}
	\item Color ingredient.
	\item Image complexity.
	\item Image order.
	\item MC metric.
	\item BL Metric.
\end{itemize}

En \cite{den2010using}, presenta una comparación de tres métricas estéticas:
\begin{itemize}
	\item Benford Law.
	\item Global Contrast Factor.
	\item Information Theory.
\end{itemize}

En \cite{den2010comparing}, presenta una comparación de cuatro métricas estéticas:
\begin{itemize}
	\item Machado and Cardoso.
	\item Ross and Ralph.
	\item Fractal Dimension.
	\item A weighted sum of the above mentioned metrics.
\end{itemize}

En \cite{den2011evolving} se presenta una aproximación multi-objetivo para arte evolutivo. Las tres funciones de fitness utilizadas son:
\begin{itemize}
	\item Benford Law.
	\item Global Contrast Factor.
	\item Ross and Ralph (bell curve).
\end{itemize}

En \cite{den2012evolving} se presenta un AE para crear arte evolutiva a partir de imágenes vectorizadas. La función de fitness utilizada es la diferencia de tono entre  distintas regiones de la imagen a distintas resoluciones.

En \cite{dipaola2009incorporating} they present an automatic fitness function specific to portrait painting based in four scores:
\begin{itemize}
	\item Resemblance.
	\item Composition (face vs background).
	\item Tonality.
	\item Color.
\end{itemize}


\section{Genetic Operators}\label{go}
\subsection{Representation}\label{go:repre}
\subsection{Initialization}\label{go:init}
\subsection{Mutation}\label{go:mutation}
\subsection{Crossover}\label{go:crossover}
\subsection{Fitness Functions}\label{go:fitness}
\subsubsection{Histogram}\label{go:fitness:hist}
HISTOGRAMA DEF: a graphical representation of the tonal distribution in an image.
\subsubsection{Image Matching}\label{go:fitness:image_match}

\section{Experimental Results} \label{exper}

\section{Conclusions and Future Work}\label{conclusions}

\subsubsection*{Aknowledments.} 

\bibliographystyle{plain}

\bibliography{evo_art}

\end{document}  