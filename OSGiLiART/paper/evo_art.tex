% !TEX encoding = UTF-8 Unicode
\documentclass[conference]{IEEEtran}
\usepackage{amsmath} 
\hyphenation{op-tical net-works semi-conduc-tor}

                                         % Activate to display a given date or no date

\begin{document}

\title{Bare Demo of IEEEtran.cls for Conferences}
% author names and affiliations
% use a multiple column layout for up to three different
% affiliations
\author{\IEEEauthorblockN{Author 1}
\IEEEauthorblockA{Affiliation 1\\
University of Granada\\
Email:}
\and
\IEEEauthorblockN{Author 2}
\IEEEauthorblockA{Affiliation 2\\
University of Granada\\
Email:}
\and
\IEEEauthorblockN{Author 3}
\IEEEauthorblockA{Affiliation 3\\
University of Granada\\
Email:}
}

\maketitle

\begin{abstract}
<Text of the summary of your article>
\end{abstract}

\section{Introduction}\label{intro}
Evolutionary art blah, blah, blah, ...

The main goal of this paper is ... We show ...

This paper is organized as follows: in Section \ref{evo_art}, a brief review on Evolutionary Art is presented. The methodology and experiments are presented in Sections \ref{go} and \ref{exper}, respectively. Finally, the conclusions and future work can be found in Section \ref{conclusions}.

\section{Evolutionary Art}\label{evo_art}

Computational Aesthetics ``is the research of computational methods that can make applicable aesthetics decisions in a similar fashion as humans can'' \cite{COMPAESTH05:13-18:2005}. In the field of computational aesthetics, evolutionary systems can play an important role, enabling the evolution of aesthetically pleasing or innovative structures \cite{dipaola2009incorporating}.

\subsection{Art Representation for Evolutive Art} \label{evo_art:repr}
\begin{itemize}
	\item Symbolic expression. The genotype is a tree of expressions and the phenotype consists in the image produced  by the evaluation of the tree.
	\item Grammars. A shape grammar is used as a formal description of the image.
	\item Using existing images as a source. 
	\item Others, such as fractals or cellular automata.
\end{itemize}
%\subsection{}

\subsection{Aesthetic measures for evolutive art}\label{evo_art:aesth}
One of the main challenges in Evolutionary Art is how to measure aesthetic value of an piece of evolutive art.

{\bf Definition} Two modes of aesthetics measures can be defined \cite{galanter2012computational}: 
\begin{enumerate}
\item ``{\em Aesthetics evaluations are expected to simulate, predict or cater to humans notions of beauty and taste.}'' This will be the definition used in this paper. 
\item ``{\em Is an aspect of meta-aesthetic exploration and usually involves aesthetic standards created by software agents in artificial worlds.}''
\end{enumerate}

According to Galanter \cite{galanter2012computational}, computational aesthetics measures can be classified in the following categories:
\begin{itemize}
	\item Based on Formulaic and Geometric Theories. The aesthetics of a piece of art are evaluated using a formula o principle (e.g., pythagorean proportions).
	\item Based in Design Principles. Like the rule of thirds or theory of color (e.g., using opposite colors).
	\item Based in Neural Networks and Connective Models. 
	\item Complexity Based Models. 
	\item Based in Evolutionary Systems:
		\begin{itemize}
			\item Interactive Evolutionary Computation. The fitness of the individuals is determined by human agents.
			\item Performance based goals. Certain properties of the art piece are evaluated and optimized based in performance measures (e.g., usable surface in a furniture design generator).  
			\item Error relative to Exemplars. The individual fitness is measured using a real-world example (e.g., a photography).
			\item Complexity measures. This type of measures is based in the idea the complexity is directly related to aesthetics and follows the path firstly stablished by Birkhoff \cite{birkhoff2003aesthetic}.
			\item Multi-objective. Given the multidimensional nature of aesthetics judgement, multi-objective EAs are a clear option in order to deal with this multidimensionality.
			\item Extensions to EA (such as, coevolution, agent swarm behavior, etc.).
		\end{itemize}
\end{itemize}

A brief classification of the aesthetic measures found in a short review can be fount in Table~\ref{table_class}.

%PAPERS LEÍDOS
%En \cite{li2012investigating}, Li et al. proponen las siguiente métricas para el aprendizaje estético:
%\begin{itemize}
%	\item Color ingredient.
%	\item Image complexity.
%	\item Image order.
%	\item MC metric.
%	\item BL Metric.
%\end{itemize}
%
%En \cite{den2010using}, presenta una comparación de tres métricas estéticas:
%\begin{itemize}
%	\item Benford Law.
%	\item Global Contrast Factor.
%	\item Information Theory.
%\end{itemize}
%
%En \cite{den2010comparing}, presenta una comparación de cuatro métricas estéticas:
%\begin{itemize}
%	\item Machado and Cardoso.
%	\item Ross and Ralph.
%	\item Fractal Dimension.
%	\item A weighted sum of the above mentioned metrics.
%\end{itemize}
%
%En \cite{den2011evolving} se presenta una aproximación multi-objetivo para arte evolutivo. Las tres funciones de fitness utilizadas son:
%\begin{itemize}
%	\item Benford Law.
%	\item Global Contrast Factor.
%	\item Ross and Ralph (bell curve).
%\end{itemize}
%
%En \cite{den2012evolving} se presenta un AE para crear arte evolutiva a partir de imágenes vectorizadas. La función de fitness utilizada es la diferencia de tono entre  distintas regiones de la imagen a distintas resoluciones.
%
%En \cite{dipaola2009incorporating} they present an automatic fitness function specific to portrait painting based in four scores:
%\begin{itemize}
%	\item Resemblance.
%	\item Composition (face vs background).
%	\item Tonality.
%	\item Color.
%\end{itemize}

%\rowcolors{2}{gray!25}{white}
\begin{table*}[!t] 
\renewcommand{\arraystretch}{1.3} 
\caption{Classification of the aesthetic measures used in a brief review of the literature on evolutive art.} 
\label{table_class} 
\centering
\begin{tabular}{|l|l|}
\hline
Type & Aesthetic Measure \\ \hline
Formulaic and Geometric Theories & Fractal dimension \cite{den2010comparing}, Image order \cite{li2012investigating}, Benford Law \cite{del2005benford}\\ \hline
Based in Design Principles &  Color contrast (hue) \cite{den2012evolving},  Color ingredient \cite{li2012investigating}, Composition, tonality and color \cite{dipaola2009incorporating}.\\ \hline
Interactive Evolutionary Computation & The electric sheep project \cite{draves2006electric} \\ \hline
Error relative to Exemplars &  Resemblance score \cite{dipaola2009incorporating}, pixel comparation \cite{aguilar2008robotic}\\ \hline
Performance based goals & Evolving virtual creatures \cite{sims1994evolving} \\\hline
Complexity measures & Image complexity \cite{li2012investigating}, Machado and Cardoso aesthetic measure \cite{machado1998computing}\\ \hline
\end{tabular}
\end{table*}

\section{Genetic Operators}\label{go}
In this section we will describe the genetic operators ...
\subsection{Representation}\label{go:repre}
Genotype -> list of shapes
\subsection{Initialization}\label{go:init}
??
\subsection{Mutation}\label{go:mutation}
??
\subsection{Crossover}\label{go:crossover}
??
\subsection{Fitness Functions}\label{go:fitness} 
For this piece of research, we focused on two measures of aesthetics: basic histogram comparison and image matching. The fitness functions are included in the ``Error relative to Exemplars'' category, using Galanter \cite{galanter2012computational} classification.
\subsubsection{Histogram comparision}\label{go:fitness:hist}
An histogram is a graphical representation of the tonal distribution in an image. The histogram for the property $i$ is computed following (\ref{eq:histogram}).


\begin{eqnarray}
	\label{eq:fitness}
	H(c, prop) = \frac{1}{N}\sum_{j=0}^N \left\{\begin{matrix}
1 & prop(j) = c\\ 
0 & otherwise
\end{matrix}\right. \\
	diff(h_1, h_2) = \sum_{j=0}^{255} |h_1(j) - h_2(j)| \\
	d_R(i) = diff(H(i, RED), H(target, RED))\\
	d_G(i) = diff(H(i, GREEN), H(target, GREEN))\\
	d_B(i) =  diff(H(i, BLUE), H(target, BLUE))\\
	fitness_{RGB}(i) = 1 - 128\frac{d_R(i) + d_G(i) + d_B(i)}{3} \\
	d_H(i) = diff(H(i, HUE), H(target, HUE))\\
	d_S(i) = diff(H(i, SAT), H(target, SAT))\\
	d_V(i) =  diff(H(i, VAL), H(target, VAL))\\
	fitness_{HSV}(i) = 1 - 128\frac{d_H(i) + d_S(i) + d_V(i)}{3}\\	
	fitness_{AVERAGE}(i) = \frac{fitness_{RGB}+fitness_{HSV}}{2}
\end{eqnarray}

\subsubsection{Image Matching}\label{go:fitness:image_match}

\section{Experimental Results} \label{exper}

\section{Conclusions and Future Work}\label{conclusions}
This paper introduces a ...

The future work for this research work includes ...
\subsubsection*{Aknowledments.} 

\bibliographystyle{IEEEtran}
\bibliography{evo_art}

\end{document}  